\documentclass{article}


% if you need to pass options to natbib, use, e.g.:
%     \PassOptionsToPackage{numbers, compress}{natbib}
% before loading neurips_2022


% ready for submission
% \usepackage{neurips_2022}


% to compile a preprint version, e.g., for submission to arXiv, add add the
% [preprint] option:
%     \usepackage[preprint]{neurips_2022}


% to compile a camera-ready version, add the [final] option, e.g.:
    \usepackage[final]{neurips_2022}


% to avoid loading the natbib package, add option nonatbib:
%    \usepackage[nonatbib]{neurips_2022}


\usepackage[utf8]{inputenc} % allow utf-8 input
\usepackage[T1]{fontenc}    % use 8-bit T1 fonts
\usepackage{hyperref}       % hyperlinks
\usepackage{url}            % simple URL typesetting
\usepackage{booktabs}       % professional-quality tables
\usepackage{amsfonts}       % blackboard math symbols
\usepackage{nicefrac}       % compact symbols for 1/2, etc.
\usepackage{microtype}      % microtypography
\usepackage{xcolor}         % colors
\usepackage{color,soul}
\usepackage{hyperref}


\title{Template for Final Project Write-up}


% The \author macro works with any number of authors. There are two commands
% used to separate the names and addresses of multiple authors: \And and \AND.
%
% Using \And between authors leaves it to LaTeX to determine where to break the
% lines. Using \AND forces a line break at that point. So, if LaTeX puts 3 of 4
% authors names on the first line, and the last on the second line, try using
% \AND instead of \And before the third author name.


\author{%
  CSCE 421 - 500 - Spring 2023 - Mortazavi
}


\begin{document}


\maketitle


\begin{abstract}
  Fill in the abstract with the a summary of your findings. The abstract should be no longer than 6-8 sentences. To communicate your ideas properly, your abstract should contain the following: 1) An introduction to the project 2) A description of the data and the task your trying to solve 3) The method you implemented to solve the problem 4) The results of your method 5) Brief conclusion.
\end{abstract}


\section{Introduction}

In this section you need to introduce the problem at a high level. What are we doing and why? Then you need to describe what you did to solve this problem. Finally, list the results of your method. If you found other resources solving similar problems, this would be the place to cite and discuss them for knowledge and comparison.

\section{Your Method}

Explain your method in detail. You must include the subsections below. A reader should be able to read this section and re-implement your design without needing your code. Give a high level description then use subsections to provide details of the specific blocks. We have provided some suggested subsections below that you might like to use.

\subsection{Data Preprocessing}

The data is available on the course Kaggle page \\\href{https://www.kaggle.com/t/2312ca7bca8f4476ace7b4cb9525e66d}{https://www.kaggle.com/t/2312ca7bca8f4476ace7b4cb9525e66d} \\
The dataset is a subset of the eICU dataset. It contains deidentified, clinical health data for over 200,000 admissions to intensive care units across the United States, between 2014-2015. The database includes vital sign measurements corresponding to each visit occurrence. Data is collected through the Philips eICU program, a critical care telehealth program that delivers information to caregivers at the bedside. This open access demo allows researchers to ascertain whether the eICU Collaborative Research Database is suitable for their work. It includes over 2,500 unit stays selected from 20 of the larger hospitals in the eICU Collaborative Research Database.

You can do whatever you'd like with the data provided. You can normalize it, one-hot encode it, or use a subset. There are no limitations as long as you don't use additional data. In this section you must describe how you decided to preprocess the data and why?

\subsection{Model Design}

What model did you choose? Why did you choose that model? What parameter did you use for the model if they weren't tuned during your hyperparameter tuning?

\subsection{Model Training}

How did you train your model?

\subsection{Hyperparameter Tuning}

How did you tune your hyperparameters to get your results?

\section{Results}

Your results should include tables and figures that compare your method to others. An example of another method would be my baseline that I've provided you with. To evaluate your model, you will download the evaluation dataset from \href{https://www.kaggle.com/competitions/csce-421-machine-learning-spring-2023-section-501/data}{here}. You will use this data as the input to your trained model. Record the outputs of the evaluation dataset and submit these results on kaggle. Your models will be submitted on the class Kaggle competition and evaluated using a held out dataset. We will use AUC-ROC to measure the fit of your model on our held out dataset. If you do better than my baseline, you get an automatic 80$\%$ on the final.

\section{Conclusion}

During this process, you've made certain design decisions. For example, your model choice or your preprocessing. Now that you have your results, describe why you believe your model did well or didn't well. Tell us what you would do if you wanted to improve the results of your method.


\end{document}
